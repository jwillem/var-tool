\documentclass[10pt,a4paper]{report}
\usepackage[utf8]{inputenc}
\usepackage[ngerman]{babel}
\usepackage{graphicx}
\usepackage{blindtext}

\graphicspath{ {images/} }

\begin{document}
  \begin{titlepage}
    {\includegraphics{logo.pdf}}\par
    \vspace{4cm}
    \centering
    {\scshape\LARGE Fakultät für Informatik \\ Hochschule Mannheim\par}
    \vspace{1cm}
    {\scshape\Large Studienarbeit\par}
    \vspace{.5cm}
    {\huge\bfseries Virtualisierte Arbeitsumgebung\break für den Test verteilter Systeme\par}

    {\LARGE Microservice-Architektur mit Docker, Kafka, Clojure \& Elm\par}
    \vspace{1cm}
    {\Large\itshape Jan-Philipp Willem\par}
    \vspace{.5cm}
    {\Large im Sommersemster 2017}
    \vfill
  \end{titlepage}

  \begin{abstract}
    \blindtext
  \end{abstract}

  \tableofcontents
  \clearpage

  \chapter{Problemstellung}
  \section{Bisheriger Ablauf}
  \section{Anforderungen}

  \chapter{Grundlagen}
  \section{Funktionale Programmierung}
  \section{Verwendete Programmiersprachen}
  \subsection{Clojure}
  \subsection{Elm}
  \section{Eingesetzte Webtechnologien}
  \subsection{Websockets}
  \subsection{Single-Page-Applications}
  \section{Microservices}
  \section{Container-Virtualisierung mit Docker}
  Docker ist eine Software zum Deployment von Applikationen innerhalb von Containern.
  Dabei ähnelt die Vorgehensweise der von Virtuellen Maschienen(VM).
  Im Vergleich zu VMs, laufen die Prozesse der Container jedoch direkt auf dem Host-Betriebssystem.
  Trotz dieser Tatsache sind die Prozesse mithilfe von unter Anderem Control-Groups und Kernel Namespaces voneinander isoliert.
  Weiterhin hat man die Möglichkeit mit gängigen Mandatory-Access-Control(MAC) -Frameworks wie SE-Linux oder App-Armor die Rechte innerhalb der Container zu beschränken.
  Als Anforderung besteht keine spezifische Hardware-Infrastruktur wie bei beispielsweise VMWare ESXi.
  Ebenso ist Docker mittlerweile auf allen gängigen Betriebssystemen lauffähig, wobei Linux-Derivate die beste Unterstützung erhalten.
  \par
  Ein Container wird mithilfe eines Dockerfiles beschrieben.
  Darin werden deskriptive Instruktionen definiert um Abhängigkeiten zu installieren, Konfigurationen vorzunehmen und andere Build-Steps auszuführen.
  Ein spezifischer Container wird als Image bezeichnet und setzt sich aus granularen Sub-Images zusammen.
  Beim Erzeugen von Images eigener Dockerfiles müssen im Vergleich zu VMs keine großen Dateien transferiert werden, da sie aus ihrem Rezept reproduzierbar sind. 
  Es besteht auch die Möglichkeit, von anderen Dockerfiles zu erben und diese über das Netzwerk verteilt bereitzustellen.
  Falls die über eine Docker-Registry angebotenen Schichten eines Containers noch nicht auf dem eigenen Rechner vorhanden sind, so werden diese heruntergeladen.
  \section{Apache Kafka}

  \chapter{VAR-Tool}
  % \section{Analyse}
  % \subsection{As-Is}
  % \subsection{To-Be}
  \section{UI-Mockup}
  \section{Architektur}
  \subsection{Geschäftsprozesse}
  \subsection{Deployment}
  \section{Umsetzung}
  \subsection{Backend}
  \subsection{Frontend}
  \section{Installation}

  \chapter{Fazit}

%   \begin{appendices}
%     \chapter{Some Appendix}
%     The contents...
%   \end{appendices}

\end{document}
