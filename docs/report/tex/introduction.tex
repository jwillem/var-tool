\chapter{Problemstellung}
Diese Arbeit unterteilt sich in vier Kapitel.
Das erste Kapitel soll eine Einleitung in die Problematik geben, welche erforscht werden soll.
Als nächstes folgen Grundlagen, welche die Techniken und Technologien des \nameref{ch:main-matter}'s erklären.
Im dritten Kapitel wird der Planungsprozess und mit dessen Hilfe der praktische Teil der Studienarbeit erläutert.
Weiterhin folgt ein Fazit der geleisteten Vorgänge und ein Ausblick, wie das Projekt weitergeführt werden könnte.

\section{Bisheriger Ablauf}
Der typische Ablauf bei Programmieraufgaben in einer Laborübung von  \ac{VAR} ist folgendermaßen:
Ein Student soll verteilte Aufgaben programmieren, welche es nötig machen, zunächst eine passende lokale Entwicklungsumgebung zu installieren.
Dazu wird die Nutzung von schwergewichtigen \acp{IDE} wie Eclipse oder Netbeans vorgeschlagen.
Ein Grund dafür sind die schon vorhandenen \ac{IDE}-Integrationen von Servern und Technologien, welche eingesetzt werden sollen.
Dennoch führt schon dieser Prozess bei einigen Studenten zu Problemen.
\par
Gelingt es die Integrationen einzurichten, so kann die Funktionsweise der eigenen Programme getestet werden.
Allerdings führt diese Herangehensweise zu keiner echten Verteilung, da alle Services auf der gleichen Maschine ausgeführt werden.
Das Ergebnis kann sich bei einer Trennung der Instanzen auf Netzwerkebene erheblich unterscheiden, oder ist unter Umständen nicht lauffähig.
\section{Anforderungen}
Es soll ein Prototyp entwickelt werden, mit dessen Hilfe ein Dozent so genannte Experimente im Kontext von verteilten Systemen definieren kann.
Diese sollen jeweils eine gegebene Anzahl an Instanzen besitzen können, auf denen Programmieraufgaben eines Studierenden und die dafür benötigten Technologien ausgeführt werden sollen.
Eine jeweilige Instanz wird in einer Konfigurationsdatei durch die Angabe eines Namens, dessen geöffneten Netzwerk-Ports und eine Argumentenliste beschrieben.
Das Deployment und die Konfiguration der Systeme soll mithilfe von Docker realisiert werden.
Auf Netzwerkebene sollen sich die Instanzen gegenseitig erreichen können, jedoch soll der Zugriff auf Instanzen anderer Nutzer und dem Internet unterbunden werden.
\par
Die Durchführung eines Experiments geschieht durch die Lösung von gewissen Aufgaben innerhalb des Instanzkontextes.
Dies geschieht durch einzelnes Hochladen der Programmpakete (Jar, War,..) pro Instanz und der anschließenden Angabe einer Main-Class und einer Argumentenliste.
Anschließend kann eine Instanz gestartet und dessen Ausgaben beobachtet werden.
\par
Der Grund für das getrennte Hochladen ist in der Hoffnung begründet, dass die echte Verteilung der Instanzen so von den Studenten besser verstanden werden kann.

