\chapter{Problemstellung}
Diese Arbeit unterteilt sich in vier Kapitel.
Das erste Kapitel soll eine Einleitung in die Problematik geben, welche erforscht werden soll.
Als nächstes folgen Grundlagen, welche die Techniken und Technologien des \nameref{ch:main-matter}'s erklären.
Im dritten Kapitel wird der Planungsprozess und mit dessen Hilfe der praktische Teil der Studienarbeit erläutert.
Weiterhin folgt ein Fazit der geleisteten Vorgänge und ein Ausblick, wie das Projekt weitergeführt werden könnte.

\section{Bisheriger Ablauf}
Der typische Ablauf bei Programmieraufgaben in der Vorlesung \ac{VAR} ist folgendermaßen:
Ein Student soll verteilte Aufgaben programmieren, welche es nötig machen, zunächst eine passende lokale Entwicklungsumgebung zu installieren.
Dazu wird die Nutzung von schwergewichtigen \acp{IDE} wie Eclipse oder Netbeans vorgeschlagen.
Grund dafür sind die schon vorhandenen \ac{IDE}-Integrationen von diversen Serven und Technologien, welche eingesetzt werden sollen.
Dennoch führt schon dieser Prozess bei einigen Studenten zu Problemen.
\par
Gelingt es die Integrationen einzurichten, so kann die Funktionsweise der eigenen Programme getestet werden.
Allerdings führt diese Herangehensweise zu keiner echten Verteilung, da alle Services auf der gleichen Maschine ausgeführt werden.
Das Ergebnis kann sich bei einer Trennung der Services auf Netzwerkebene erheblich unterscheiden oder ist unter Umständen gar nicht lauffähig.
\par
[TODO: Beispiel einer Aufgabe z.B. RMI-Chat]
\section{Anforderungen}
Ziel soll es sein, dass ein Dozent Experimente definieren kann, welche eine gegebene Anzahl an Instanzen besitzen.
Eine jeweilige Instanz wird durch die Angabe eines Namens, dessen geöffneten Netzwerk-Ports und eine Argumentenliste beschrieben.
Die Konfiguration der Systeme soll mit Hilfe eines Dockerfiles realisiert werden.
Die Instanzen sollen sich gegenseitig erreichen können, jedoch soll der Zugriff auf Instanzen anderer Nutzer und dem Internet unterbunden werden.
\par
Weiterhin sollen von einem Studenten gewisse Aufgaben innerhalb dieser Experimente gelöst werden.
Dies geschieht durch einzelnes Hochladen der Programmpakete (Jar, War,..) pro Instanz und der Angabe einer Main-Class und einer Argumentenliste.
Anschließend kann eine Instanz gestartet werden und dessen Ausgaben beobachtet werden.
\par
Der Grund für das getrennte Hochladen ist darin begründet, dass die echte Verteilung der Instanzen so von den Studenten eventuell besser verstanden werden kann.

