\begin{abstract}
  Ein verteiltes Softwaresystem zu entwickeln und ebenso dessen Auswirkungen zu verstehen, ist kein Leichtes für einen durchschnittlichen Informatik-Studenten.
  Oft ist es der Fall, dass es schon anspruchsvoll genug ist, alleine die Grundlagen umzusetzen, welche bei einer Aufgabe gefordert sind. 
  Um diese Hürde etwas zu erleichtern, soll eine Umgebung geschaffen werden, welche es ermöglicht, eine verteilte Anwendung unabhängig des eigenen Computers zu testen.
  Eine Programmieraufgabe soll von einem Dozenten als ein Experiment definiert werden können, welches aus einer gegebenen Anzahl an Instanzen eines ebenso definierbaren Servers-Systems besteht.
  Weiterhin soll das Experiment von einem Studenten durchgeführt werden können, welches das Hochladen der eigenen Lösung und das anschließende Analysieren der darin resultierenden Ausgaben der voneinander getrennten Rechner-Instanzen darstellt.
  \\\\\\
  \hspace*{.43\textwidth}\textbf{Abstract}
  \\\\
  \blindtext
\end{abstract}
