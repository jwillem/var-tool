\begin{abstract}
  Ein verteiltes Softwaresystem zu entwickeln und ebenso dessen Auswirkungen zu verstehen, ist kein leichtes für einen durchschnittlichen Informatik-Studenten.
  Oft ist es der Fall, dass es schon anspruchsvoll genug ist, alleine die Grundlagen umzusetzen, welche es bei einer Aufgabe anzuwenden vermag. 
  Um diese Hürde etwas zu erleichtern, soll eine Umgebung geschaffen werden, welche es ermöglicht, eine verteilte Aufgabe unabhängig des eigenen Computers zu testen.
  Eine Programmieraufgabe soll von einem Dozenten als ein Experiment definiert werden können, welches eine gegebene Anzahl an Instanzen besitzt und durch einen Studenten genutzt werden kann, die Ausgaben der erstellten Programme auf voneinander getrennten Rechnern zu beobachten.
\end{abstract}
