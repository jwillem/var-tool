\chapter{Fazit}
Während der Entwicklung hatte sich herausgestellt, dass die eigentliche Überlegung, Apache Kafka für die Integration der einzelnen Services der Applikation zu nutzen, erheblich mehr Aufwand bedeutet hätte.
In der Vorbereitungsphase wurde eine Bibliothek Namens \textit{kafka-websocket} ausgewählt, welche es versprechen sollte, die Clients über Websocket an Kafka zu verbinden.
Bei näherer Betrachtung der Library wurde leider erkannt, dass das Format der einzelnen gesendeten Nachrichten an den sogenannten Kafka-Websocket-Proxy schwer zu verändern ist.
Dies passte jedoch nicht zu dem vorherig überlegten Nachrichten-Schema zur Kommunikation zwischen Server und Client.
Es wurde statt dem Einsatz von \textit{kafka-websocket} ein eigener Websocket-Endpoint im Server entwickelt, welcher das geplante Nachrichten-Schema umsetzt.
Die Übermittlung von Log-Daten aus den User-Containern in den Server wurde stattdessen über die Dateiebene des Docker-Hosts gestaltet.
Die User-Container schreiben eine \texttt{stdout}-Datei und lesen eine \texttt{stdin}-Datei um Eingaben zu akzeptieren.
Der Server liest die Datei \texttt{stdout} und gibt den Inhalt per Websocket an den Client weiter, bzw. schreibt in die Datei \texttt{stdin} um Eingaben vom Client an den Container weiterzugeben.
\par
Die Grundüberlegung bzw. in der Architektur\footnote{Hierzu kann im Repository des Projekts ein Diagramm mit dem Namen \texttt{deployment\_kafka.pdf} gefunden werden.} der Applikation Kafka einzusetzen, könnte in einer weiteren Arbeit erforscht werden.
Erzeugte Logs der User-Container könnten mit Kafka empfangen und auch persistent gespeichert werden.
So wäre es denkbar, dass die einzelnen Ergebnisse der Durchführungen der Experimente auch von einem Dozenten im Nachhinein betrachtet werden könnten.
Im Gesamten würde Kafka ebenso den Micro-Service Gedanke des Projekts weiter ausbauen.
\par
Die Verbindung zwischen Client und Server sollte in Zukunft nach aktuellen best Practices vorsorglich besser verschlüsselt werden.
Dazu kann für HTTP ein SSL-Zertifikat von \ac{bspw.} \textit{let's encrypt} genutzt werden.
Ebenso sollten die Websocket-Verbindungen mit TLS über WSS eingerichtet werden.
\par
Die geplante Abschottung der Container wurde bisher nur mit den Bordmitteln von Docker realisiert.
Diese ersetzen jedoch nicht den Einsatz von einzelnen Firewall-Rules mithilfe von \ac{bspw.} \textit{iptables}.
Bei einem tatsächlichen Deployment der Applikation könnte daher Server und Client über einen weiteren Proxy-Container (\textit{nginx}) geleitet werden, der außerdem eine Firewall implementiert.
\par
% \begin{itemize}
  % \item Umsetzung mit Kafka leider nicht m"oglich.
  % \item mehraufwand eigenen kafka proxy zu schreiben
  % \item grundüberlegung von kafka kann dennoch in folgenden arbeiten zu thema kommen
  % \item kafka kann log daten empfangen und auch dauerhaft vorhalten
  % \item microservice gedanke kann weiter ausgebaut werden
  % \item Ergebnisse (eigenes Kapitel?)
  % \item Performance!
  % \item websocket sollte über tls erfolgen
  % \item Im Experiment könnte noch bei geringem Aufand ein dnd-container zum einsatz kommen. 
  % \item Admin-Client muss noch in weiterer Arbeit entwickelt werden
  % \item Abschottung der Container muss noch mit bspw iptables geschehen
% \end{itemize}
\clearpage
